\documentclass[a4paper, USenglish, cleveref, autoref, thm-restate]{lipics-v2021}

\bibliographystyle{plainurl}

\title{Formally verifying a vertical cell decomposition algorithm}

\author{Yves Bertot}{Inria Center at Université Côte d'Azur, France}
       {yves.bertot@inria.fr}
       {https://orcid.org/0000-0001-5052-3019}{}

\authorrunning{Y. Bertot}

\Copyright{Yves Bertot}


\ccsdesc[300]{Theory of computation~Computational geometry}
\ccsdesc[500]{Theory of computation~Program verification}
\ccsdesc[500]{Theory of computation~Higher order logic}
\ccsdesc[500]{Theory of computation~Logic and verification}
\ccsdesc[500]{Theory of computation~Type theory}

\keywords{Formal Verification, Motion planning, algorithmic geometry}



%Editor-only macros:: begin (do not touch as author)%%%%%%%%%%%%%%%%%%%%%%%%%%%%%%%%%%
\EventEditors{John Q. Open and Joan R. Access}
\EventNoEds{2}
\EventLongTitle{42nd Conference on Very Important Topics (CVIT 2016)}
\EventShortTitle{CVIT 2016}
\EventAcronym{CVIT}
\EventYear{2016}
\EventDate{December 24--27, 2016}
\EventLocation{Little Whinging, United Kingdom}
\EventLogo{}
\SeriesVolume{42}
\ArticleNo{23}
%%%%%%%%%%%%%%%%%%%%%%%%%%%%%%%%%%%%%%%%%%%%%%%%%%%%%%

\begin{document}
\maketitle

\begin{abstract}
We design an succession of algorithms that decompose a finite area of the
plane into cells in such a way that none of these cells contain obstacles.
We provide a formally verified proof that this algorithm produces safe cells,
obtained using the Rocq prover and the mathematical components library.

This article discusses the assumptions made concerning the area and the
obstacles, the structure of the main algorithm, especially the way it handles
the possibility of degenerate cases, the specification of the final safety
property, and the organisation of the formal proof.
\end{abstract}

\section{Introduction}

\end{document}

\documentclass[a4paper, USenglish, cleveref, autoref, thm-restate]{lipics-v2021}

\bibliographystyle{plainurl}

\title{Formally verifying a vertical cell decomposition algorithm}

\author{Yves Bertot}{Inria Center at Université Côte d'Azur, France}
       {yves.bertot@inria.fr}
       {https://orcid.org/0000-0001-5052-3019}{}

\author{Thomas Portet}{Inria Center at Université Côte d'Azur, France}
       {thomas.portet@inria.fr}
       {}{}


\authorrunning{Y. Bertot and T. Portet}

\Copyright{Yves Berto and Thomas Portett}


\ccsdesc[300]{Theory of computation~Computational geometry}
\ccsdesc[500]{Theory of computation~Program verification}
\ccsdesc[500]{Theory of computation~Higher order logic}
\ccsdesc[500]{Theory of computation~Logic and verification}
\ccsdesc[500]{Theory of computation~Type theory}

\keywords{Formal Verification, Motion planning, algorithmic geometry}



%Editor-only macros:: begin (do not touch as author)%%%%%%%%%%%%%%%%%%%%%%%%%%%%%%%%%%
\EventEditors{John Q. Open and Joan R. Access}
\EventNoEds{2}
\EventLongTitle{42nd Conference on Very Important Topics (CVIT 2016)}
\EventShortTitle{CVIT 2016}
\EventAcronym{CVIT}
\EventYear{2016}
\EventDate{December 24--27, 2016}
\EventLocation{Little Whinging, United Kingdom}
\EventLogo{}
\SeriesVolume{42}
\ArticleNo{23}
%%%%%%%%%%%%%%%%%%%%%%%%%%%%%%%%%%%%%%%%%%%%%%%%%%%%%%

\begin{document}
\maketitle

\begin{abstract}
We design an succession of algorithms that decompose a finite area of the
plane into cells in such a way that none of these cells contain obstacles.
We provide a formally verified proof that this algorithm produces safe cells,
obtained using the Rocq prover and the mathematical components library.

This article discusses the assumptions made concerning the area and the
obstacles, the structure of the main algorithm, especially the way it handles
the possibility of degenerate cases, the specification of the final safety
property, and the organisation of the formal proof.
\end{abstract}

\section{Introduction}

\section{The data components}
\subsection{points and edges}
Explain the choice to have two distinct approaches to describe edges,
the signed area formula, the above, below edge predicates, the point
on edge predicate.

Note that no decision is taken based on the results of a division, but
only on comparison to 0 for polynomial expressions.

Talk about the projection of a point on an edge, and the choice to
have this function be of option type.   Give the valid edge predicate.
Note that this predicate is not used in the code.
\subsection{The cells}
Each cell has a high edge, a low edge, a left side, and the right side.  For
the lateral sides, we do not only keep the the x-coordinate, but we record
the ordered sequence of points that correspond to events in contact with this
edge.

The list of points for cells where the low edge and the right edge
meet on the left side is reduced to a single point.

For open cells, the sequence of points representing the right-hand side is
left empty.

\subsection{invariants for sequences of cells}
\begin{itemize}
\item adjacent cells
\item high edges are above low edges
\item extremities of the sequence reach to the bottom and top edges
\end{itemize}
\subsection{Data invariants for cells}
Predicates for well-formedness in cells ensure that the sides are
vertical and within the vertical cylinders anchored in both the high
and low edge.  Moreover, the points are sorted vertically (in
descending order), and the extremities of the points are on the edges.

A note on extremities: these play a role in the proofs, but in
practice, they could be replaced by arbitrary approximation
points as long as they preserve the order.

\section{The scan state}

Isolating the last created cells (both open and closed), caching the last
high edge and the last x-coordinate.

\section{The phases of the algorithm}
\begin{enumerate}
\item sort obstacle edges by their left-side extremity, construct events with
outgoing edges
\item process the events in turn, maintaining a record with the order list
of currently open cells, broken down into three parts, the current set
of closed cells, the high edge of the last created open cell, and the
x-coordinate of the last processed event.
\end{enumerate}

When processing each event four cases are treated differently:
\begin{itemize}
\item If the considered event is further right than the last processed event
\item If the considered event is on the same vertical line, but above the
high edge of the last created open cell
\item If the considered event is on the same vertical line, but below the
high edge of the last created open cell
\item If the considered event is on the high edge of the last created open cell
\end{itemize}

\subsection{Sorting the sequence of events}
Using an insertion sort pattern, but when several edges have the same
left-hand side, no new event is created, instead edges are added in
the sequence of outgoing edges from an existing event.  The same holds
for right-hand extremities of events: when an event already exists for
that point, nothing is inserted.

Choice to not sort the list of outgoing edges.

\subsection{creating the first scan state}

The scan state has fields that require a first event to be processed
even before starting the main loop.  When processing this first event,
it obviously is not the right-hand side of any edge.  This obvious
fact is expressed logically by exploiting the fact that the sequence
of events is sorted.

\subsection{Work performed in the first case}
\begin{itemize}
\item Decompose the sequence of open cells into three parts
\item Create closed cells by adding a right-hand side to all cells in contact
with the event.
\item Note what the last closed and last open cells are in contact
  through their right-hand and left-hand side, they share the same
  high edge and the first two points on these sides.
\end{itemize}
\subsection{Work performed for the second case}
Same as the previous case, but the search for the cell sequence
decomposition only processes cells above the last created open cell.
\subsection{Work performed for the third case}
In the third case, the considered event cannot be the right extremity of
an active edge, so that this point has to be added to the right hand side
of the last closed cell.  No analysis of the open cells.

With respect to the open cells, two cases can happen.
\begin{itemize}
\item if the considered event has no outgoing edges, no new cells
are created, only the last open cell is modified to ensure that the considered
event is added among the points of its left-hand side.
\item if the considered event has outgoing edges, cells are created as
  in the first two cases, except that the first created cell receives
  a left side constructed by using parts of the last open cell's left
  hand side and the current point.  The new last open cell is obtained
directly as in the first two cases.
\end{itemize}
\subsection{Work performed for the fourth case}
The considered event is on the high edge of the last closed cell.  We
need to compute whether there are more cells to be closed, and the
last closed cell will be a cell that was open until know and whose
low edge finishes at the considered event.  On the side of open cells,
again two cases may occur depending on the presence of outgoing
edges.  If there are outgoing edges, new open cells are created from
these edges, using the low edge of the last open cell for the low edge
of the first newly created open cell.  Also, the sequence of 
left side points of
the first newly created open cell is exactly the sequence of left side
points of the last open cell.  The new last open cell is the last of
the newly created cells, as expected.

If there are no outgoing edges, the last open cell is updated in two
ways: the left side receives as first element the projection of the
considered event on the high edge of the last contact cell, and the
high edge is the high edge of the last contact cell.

\subsection{Specifying the behavior}
This paper describes a proof of safety, there are desirable properties
that will not be subject to a formal proof.  The safety property will
be expressed in the following manner: any point in the safe part of a
closed cell is distinct from the obstacles and the events given as input.

The safe part of a closed cell is defined in this manner:
\begin{itemize}
\item either the point x-coordinate is between the x-coordinates of
  the left-side points or the x-coordinates of the right-side points
  (with no equality), strictly above the low edge, and strictly below
  the high edge
\item or the point has the same x-coordinate as the left-side points,
  and it is distinct from any of the points in the sequence of
  left-side points,
\item or the point has the same x-coordinate as the right-side points,
  and it is distinct from any of the points in the sequence of
  right-side points.
\end{itemize}

This safety property is only guaranteed under a collection of
assumptions concerning the input segments and events:
\begin{itemize}
\item The only allowed intersections between segments are at their
  extremities
\item all events (including segment extremeties) are inside the box
 defined by the bottom and top edges.
\end{itemize}

A property that does not appear immediately as a safety property is
that the middle of every cell is strictly included in the cell.  This
property is useful for users of this algorithm who wish to use a cell
as turning space to move from one door of the cell to another door of
the same cell, when both doors are on the same side.  This property is
the main reason for having a specific treatment for degenerate cases.
This specific treatment actually guarantees that every closed cell has
a left-hand side with an x-coordinate that is lower than the
x-coordinate of the right-hand side.

To make sure that every cell contains at least a safe point, we
actually need the property that every event has a list of outgoing
edges without duplications.  This needs to be guaranteed either by the
first phase of the program (which should check that an edge is not
already present when adding it to the outgoing edges of an event), or
by the input data (for which we could specify that it contains no
duplications).

\end{document}

\documentclass[a4paper, USenglish, cleveref, autoref, thm-restate]{lipics-v2021}

\bibliographystyle{plainurl}

\title{Formally verifying a vertical cell decomposition algorithm}

\author{Yves Bertot}{Inria Center at Université Côte d'Azur, France}
       {yves.bertot@inria.fr}
       {https://orcid.org/0000-0001-5052-3019}{}

\authorrunning{Y. Bertot}

\Copyright{Yves Bertot}


\ccsdesc[300]{Theory of computation~Computational geometry}
\ccsdesc[500]{Theory of computation~Program verification}
\ccsdesc[500]{Theory of computation~Higher order logic}
\ccsdesc[500]{Theory of computation~Logic and verification}
\ccsdesc[500]{Theory of computation~Type theory}

\keywords{Formal Verification, Motion planning, algorithmic geometry}



%Editor-only macros:: begin (do not touch as author)%%%%%%%%%%%%%%%%%%%%%%%%%%%%%%%%%%
\EventEditors{John Q. Open and Joan R. Access}
\EventNoEds{2}
\EventLongTitle{42nd Conference on Very Important Topics (CVIT 2016)}
\EventShortTitle{CVIT 2016}
\EventAcronym{CVIT}
\EventYear{2016}
\EventDate{December 24--27, 2016}
\EventLocation{Little Whinging, United Kingdom}
\EventLogo{}
\SeriesVolume{42}
\ArticleNo{23}
%%%%%%%%%%%%%%%%%%%%%%%%%%%%%%%%%%%%%%%%%%%%%%%%%%%%%%

\begin{document}
\maketitle

\begin{abstract}
We design an succession of algorithms that decompose a finite area of the
plane into cells in such a way that none of these cells contain obstacles.
We provide a formally verified proof that this algorithm produces safe cells,
obtained using the Rocq prover and the mathematical components library.

This article discusses the assumptions made concerning the area and the
obstacles, the structure of the main algorithm, especially the way it handles
the possibility of degenerate cases, the specification of the final safety
property, and the organisation of the formal proof.
\end{abstract}

\section{Introduction}

\section{The data components}
\subsection{The cells}
Each cell has a high edge, a low edge, a left side, and the right side.  For
the lateral sides, we do not only keep the the x-coordinate, but we record
the ordered sequence of points that correspond events in contact with this
edge.

For open cells, the sequence of points representing the right-hand side is
left empty.

\subsection{invariants for sequences of cells}
\begin{itemize}
\item adjacent cells
\item 
\end{itemize}
\subsection{Data invariants for cells}
Predicates for

\section{The scan state}

Isolating the last created cells (both open and closed), caching the last
high edge and the last x-coordinate.

\section{The phases of the algorithm}
\begin{enumerate}
\item sort obstacle edges by their left-side extremity, construct events with
outgoing edges
\item process the events in turn, maintaining a record with the order list
of currently open cells, broken down into three parts, the current set
of closed cells, the high edge of the last created open cell, and the
x-coordinate of the last processed event.
\end{enumerate}

When processing each event four cases are treated differently:
\begin{itemize}
\item If the considered event is further right than the last processed event
\item If the considered event is on the same vertical line, but above the
high edge of the last created open cell
\item If the considered event is on the same vertical line, but below the
high edge of the last created open cell
\item If the considered event is on the high edge of the last created open cell
\end{itemize}
\subection{Work performed in the first case}
\begin{itemize}
\item Decompose the sequence of open cells into three parts
\item Create closed cells by adding a right-hand side to all cells in contact
with the event.
\item Note what the last closed and last open cell are.
\end{itemize}
\subsection{Work performed for the second case}
Same as the previous case, but the search for the cell sequence
decomposition only processes cells above the last created open cell.
\subsection{Work performed for the third case}
Two cases must be considered:
\begin{itemize}
\item if the considered event has no outgoing edges, it can only be an
extra point to be added on the left hand side of the current last created
edge and this point must also be added in the right hand side of the

\end{itemize}

\end{document}

\documentclass[a4paper, USenglish, cleveref, autoref, thm-restate]{lipics-v2021}

\bibliographystyle{plainurl}

\title{Formally verifying a vertical cell decomposition algorithm}

\author{Yves Bertot}{Inria Center at Université Côte d'Azur, France}
       {yves.bertot@inria.fr}
       {https://orcid.org/0000-0001-5052-3019}{}

\author{Thomas Portet}{Inria Center at Université Côte d'Azur, France}
       {thomas.portet@inria.fr}
       {}{}


\authorrunning{Y. Bertot and T. Portet}

\Copyright{Yves Berto and Thomas Portett}


\ccsdesc[300]{Theory of computation~Computational geometry}
\ccsdesc[500]{Theory of computation~Program verification}
\ccsdesc[500]{Theory of computation~Higher order logic}
\ccsdesc[500]{Theory of computation~Logic and verification}
\ccsdesc[500]{Theory of computation~Type theory}

\keywords{Formal Verification, Motion planning, algorithmic geometry}



%Editor-only macros:: begin (do not touch as author)%%%%%%%%%%%%%%%%%%%%%%%%%%%%%%%%%%
\EventEditors{John Q. Open and Joan R. Access}
\EventNoEds{2}
\EventLongTitle{42nd Conference on Very Important Topics (CVIT 2016)}
\EventShortTitle{CVIT 2016}
\EventAcronym{CVIT}
\EventYear{2016}
\EventDate{December 24--27, 2016}
\EventLocation{Little Whinging, United Kingdom}
\EventLogo{}
\SeriesVolume{42}
\ArticleNo{23}
%%%%%%%%%%%%%%%%%%%%%%%%%%%%%%%%%%%%%%%%%%%%%%%%%%%%%%

\begin{document}
\maketitle

\begin{abstract}
We design an succession of algorithms that decompose a finite area of the
plane into cells in such a way that none of these cells contain obstacles.
We provide a formally verified proof that this algorithm produces safe cells,
obtained using the Rocq prover and the mathematical components library.

This article discusses the assumptions made concerning the area and the
obstacles, the structure of the main algorithm, especially the way it handles
the possibility of degenerate cases, the specification of the final safety
property, and the organisation of the formal proof.
\end{abstract}

\section{Introduction}

\section{The data components}
\subsection{points and edges}
Explain the choice to have two distinct approaches to describe edges,
the signed area formula, the above, below edge predicates, the point
on edge predicate.

Note that no decision is taken based on the results of a division, but
only on comparison to 0 for polynomial expressions.

Talk about the projection of a point on an edge, and the choice to
have this function be of option type.   Give the valid edge predicate.
Note that this predicate is not used in the code.
\subsection{The cells}
Each cell has a high edge, a low edge, a left side, and the right side.  For
the lateral sides, we do not only keep the the x-coordinate, but we record
the ordered sequence of points that correspond to events in contact with this
edge.

The list of points for cells where the low edge and the right edge
meet on the left side is reduced to a single point.

For open cells, the sequence of points representing the right-hand side is
left empty.

\subsection{invariants for sequences of cells}
\begin{itemize}
\item adjacent cells
\item high edges are above low edges
\item extremities of the sequence reach to the bottom and top edges
\end{itemize}
\subsection{Data invariants for cells}
Predicates for well-formedness in cells ensure that the sides are
vertical and within the vertical cylinders anchored in both the high
and low edge.  Moreover, the points are sorted vertically (in
descending order), and the extremities of the points are on the edges.

A note on extremities: these play a role in the proofs, but in
practice, they could be replaced by arbitrary approximation
points as long as they preserve the order.

\section{The scan state}

Isolating the last created cells (both open and closed), caching the last
high edge and the last x-coordinate.

\section{The phases of the algorithm}
\begin{enumerate}
\item sort obstacle edges by their left-side extremity, construct events with
outgoing edges
\item process the events in turn, maintaining a record with the order list
of currently open cells, broken down into three parts, the current set
of closed cells, the high edge of the last created open cell, and the
x-coordinate of the last processed event.
\end{enumerate}

When processing each event four cases are treated differently:
\begin{itemize}
\item If the considered event is further right than the last processed event
\item If the considered event is on the same vertical line, but above the
high edge of the last created open cell
\item If the considered event is on the same vertical line, but below the
high edge of the last created open cell
\item If the considered event is on the high edge of the last created open cell
\end{itemize}

\subsection{Sorting the sequence of events}
Using an insertion sort pattern, but when several edges have the same
left-hand side, no new event is created, instead edges are added in
the sequence of outgoing edges from an existing event.  The same holds
for right-hand extremities of events: when an event already exists for
that point, nothing is inserted.

Choice to not sort the list of outgoing edges.

\subsection{creating the first scan state}

The scan state has fields that require a first event to be processed
even before starting the main loop.  When processing this first event,
it obviously is not the right-hand side of any edge.  This obvious
fact is expressed logically by exploiting the fact that the sequence
of events is sorted.

\subsection{Work performed in the first case}
\begin{itemize}
\item Decompose the sequence of open cells into three parts
\item Create closed cells by adding a right-hand side to all cells in contact
with the event.
\item Note what the last closed and last open cells are in contact
  through their right-hand and left-hand side, they share the same
  high edge and the first two points on these sides.
\end{itemize}
\subsection{Work performed for the second case}
Same as the previous case, but the search for the cell sequence
decomposition only processes cells above the last created open cell.
\subsection{Work performed for the third case}
In the third case, the considered event cannot be the right extremity of
an active edge, so that this point has to be added to the right hand side
of the last closed cell.  No analysis of the open cells.

With respect to the open cells, two cases can happen.
\begin{itemize}
\item if the considered event has no outgoing edges, no new cells
are created, only the last open cell is modified to ensure that the considered
event is added among the points of its left-hand side.
\item if the considered event has outgoing edges, cells are created as
  in the first two cases, except that the first created cell receives
  a left side constructed by using parts of the last open cell's left
  hand side and the current point.  The new last open cell is obtained
directly as in the first two cases.
\end{itemize}
\subsection{Work performed for the fourth case}
The considered event is on the high edge of the last closed cell.  We
need to compute whether there are more cells to be closed, and the
last closed cell will be a cell that was open until know and whose
low edge finishes at the considered event.  On the side of open cells,
again two cases may occur depending on the presence of outgoing
edges.  If there are outgoing edges, new open cells are created from
these edges, using the low edge of the last open cell for the low edge
of the first newly created open cell.  Also, the sequence of 
left side points of
the first newly created open cell is exactly the sequence of left side
points of the last open cell.  The new last open cell is the last of
the newly created cells, as expected.

If there are no outgoing edges, the last open cell is updated in two
ways: the left side receives as first element the projection of the
considered event on the high edge of the last contact cell, and the
high edge is the high edge of the last contact cell.

\subsection{Specifying the behavior}
This paper describes a proof of safety, there are desirable properties
that will not be subject to a formal proof.  The safety property will
be expressed in the following manner: any point in the safe part of a
closed cell is distinct from the obstacles and the events given as input.

The safe part of a closed cell is defined in this manner:
\begin{itemize}
\item either the point x-coordinate is between the x-coordinates of
  the left-side points or the x-coordinates of the right-side points
  (with no equality), strictly above the low edge, and strictly below
  the high edge
\item or the point has the same x-coordinate as the left-side points,
  and it is distinct from any of the points in the sequence of
  left-side points,
\item or the point has the same x-coordinate as the right-side points,
  and it is distinct from any of the points in the sequence of
  right-side points.
\end{itemize}

In a similar manner, we can describe unsafe locations as points lying
on an obstacle:
\begin{itemize}
\item the x-coordinate of these points is between the x-coordinates of
the two extremities,
\item the signed area function for these points and the edge
  extremities has value 0,
\item any point taken from the list of input events is unsafe.
\end{itemize}

is only guaranteed under a collection of
assumptions concerning the input segments and events:
\begin{itemize}
\item The only allowed intersections between segments are at their
  extremities
\item all events (including segment extremeties) are inside the box
 defined by the bottom and top edges.
\end{itemize}

In the end, the main safety property can be expressed as a very simple
set-theoretical statement:  The intersection between the set of unsafe
locations and the safe part of all produced cells is empty.

A property that does not appear immediately as a safety property is
that the middle of every cell is strictly included in the cell.  This
property is useful for users of this algorithm who wish to use a cell
as turning space to move from one door of the cell to another door of
the same cell, when both doors are on the same side.  This property is
the main reason for having a specific treatment for degenerate cases.
This specific treatment actually guarantees that every closed cell has
a left-hand side with an x-coordinate that is lower than the
x-coordinate of the right-hand side.

To make sure that every cell contains at least a safe point, we
actually need the property that every event has a list of outgoing
edges without duplications.  Our code to generate the sequence of
events does not include steps to guarantee this, but if the input
list of obstacles has no duplications, the code does not duplicate them.

\section{Proving the safety property}
\subsection{Mirroring contexts}
The infra-structure of definitions provided by the Mathematical
Components library is not fully compatible with the extraction
process.  To obtain code that is amenable for extraction, we write a
first file called {\tt generic\_trajectories}, that does not rely on
any of the mathematical structures present in math comp, but is
parameterized by a given type supposed to represent numbers, with 4
operations (addition, subtraction, multiplication, division) and two
comparisons.  Also, this algorithm relies on a degraded type of edges,
where the condition that the first point is on the left of the second
point is absent from the data-structure.

The generic algorithm is used in all the proof files.  In each of
these proof files, we open a section where the type of numbers is
actually provided by a structure of type {\tt realFieldType}.  It
should be noted that all the proofs performed about this algorithm
only rely on the fact that the type of numbers forms a field, so that
any implementation of rational numbers is for making all the
computations.

Every function imported from {\tt generic\_trajectories} is a function
with at least 7 arguments.  To make the whole setup comfortable, a
header of notation definition is provided to instantiate the function
to the type and operations provided by the field structure.


\subsection{Main organization of the proof}
The first part of the algorithm consists in preparing the list of
events, processing these events in the order given by the list
simulates the intuitive behavior of moving a vertical line
progressively from the left-hand side of the box to the right-hand
side.

The algorithm maintains a sequence of open cells between two events,
the topological properties of these cells are preserved as one moves
from one event to the next: the low and high edges of the cell are the
same, the neighbor cells on the other side of these edges are the
same, etc.  Only when an event occurs does the structure of the list
of open cells change.  Some cells that were present in the list of
open cells are removed from that list, and after some modification,
they are added to the list of closed cells, new cells are created
where the low or high edge, or both, are taken from the outgoing edges
of the currently processed event.

This feels like reasoning by homotopy.  Depending on the position of
the sweeping line, the intersections between these lines and the low
and high edges of each open cell are at different heights, but these
points are always ordered on this line in the same manner.  In other
words, the topological structure of the sequence of cells evolves
continuously as the sweeping line moves from one event to the other.
This relies on a strong invariant concerning the sequence of events:
any change of topology (for example the end of an active edge) must be
recorded in the sequence of events.  This is reflected by two
predicates {\tt close\_alive\_edges} and {\tt close\_edges\_from\_events}.

In our work, we rely quite a lot on a relation between edges, named
{\em edge\_below} and noted by the infix symbol {\tt <|}.  It turns
out that this relation is reflexive, not transitive, and not
antisymmetric, so many of the tools we
would like to inherit from the ambient formal mathematics library
cannot be used directly out of the box.

In the proof, we established five levels of invariants:
\begin{enumerate}
\item In the first level, we state properties that concern only the
sequence placement of low and high edges for the open cells,
\item In the second level, we add the cache consistency properties of
  the scan state, the expected properties of the sequence of events,
  and the consistency between the open cells and the future events,
\item In the third level, we add properties that are specific to the
  last created cell,
\item In the third level, we add mainly the properties that concern
  the closed cells that were built so far and their interactions with
  the current open cells.  The main purpose of the properties
  available in this invariant is to contribute to the proof that
  all closed cells are disjoint and non-empty.
\item In the fourth level, we add mainly the properties that concern
  processed edges and their coverage by existing closed and open
  cells.
\item In the fiftth level, we add mainly the properties that concern
  the points recorded in cell sides.  The point is to show that cell
  sides enumerate the points that are unsafe on the vertical
  sides of cells.
\end{enumerate}
In the end, the safety property for the space inside a given cell was
guaranteed by the main argument that if edges are covered by the high
edges of all closed cells, and all closed cells are disjoint, then
edges cannot appear in the interior of a given cell.

We performed two proofs of the main result.  The first proof takes
stronger assumptions on the inputs: it assumes that two events are
never vertically aligned.  If this is the case, no specific treatement
is needed to handle the last created open cell, which is guaranteed to
be transformed later into a non-empty closed cell.  The second proof
observes the case where verical alignment of events is permitted.  In
this case, it sometimes necessary to not create new open cells, but
only to modify existing cells.n
\subsection{Algebraic properties of the area function}
The area function is instrumental to detect when a point is above or
below an edge.  More precisely, it makes it possible to detect when a
point is above the straight line that carries the obstacle.

This area function is given by a determinant formula that we learned
from Knuth, but is probably part of the folklore in algorithmic
geometry.  It is quite important that this is only computed using a
determinant, as it only needs ring operations (no division), without
ever mentioning the projection of the given point on the given
straight line.

For proofs, however it is useful to have an alternative point of view,
where one expresses the property of being above using a comparison of
the second coordinates of the given point with the projected point on
the given line.

The edge comparison relation is directly related with the area
function.  To detect whether an edge is below another, we simply
verify whether the two extremities of one are in the same half plane
with respect to the other.

In the end, to express that a point lies on a segment, we use the area
function to detect that the point is aligned with the extremities, and
using the invariant that no obstacle is given is vertical, we simply
verify that the first coordinate is between the first coordinates of
the segment extremities.
\subsection{Building up invariants}
\subsection{The key property of disjoint cells}
In the end, each closed cell has four boundaries: two vertical ones, a
low boundary (a fragment of an obstacle that is not vertical) and a
high boundary (again, not vertical).  We define three zones:
\begin{itemize}
\item inside the union of the cell and its boundaries,
\item strictly inside the cell, not on the boundaries,
\item half-inside (for lack of a better name): inside the cell and its
 boundaries, but neither on the bottom boundary nor on the left boundary.
\end{itemize}
The main key property that we want to guarantee is that every point
strictly inside a cell does not belong to any obstacle.  To establish
this, we decompose the problem in two parts: first we show
that the ``half-inside'' of all closed cells are disjoint.  Later, we
prove that every obstacle is included the union of high boundaries of
all cells, this makes it possible to conclude with the key property.

In the rest of this section, we 

To prove that closed cells are disjoint, we need to prove that new
cells are disjoint with the existing ones, when new cells are added.
At any iteration, the newly created closed cell actually come from
existing open cells.  Therefore, we need to prove the invariant that
closed cells are disjoint 
\end{document}

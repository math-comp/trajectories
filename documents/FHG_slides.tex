\documentclass[compress]{beamer}
\usepackage[latin1]{inputenc}
\usepackage{alltt}
\newdimen\topcrop
\topcrop=10cm  %alternatively 8cm if the pdf inclusions are in letter format
\newdimen\topcropBezier
\topcropBezier=19cm %alternatively 16cm if the inclusions are in letter format

\setbeamertemplate{footline}[frame number]
\title{Smooth trajectories in straight line mazes}
\author{Yves Bertot\\
Joint work with Thomas Portet, Quentin Vermande}
\date{April 2023}
\mode<presentation>
\begin{document}

\maketitle
\begin{frame}
\frametitle{The game}
\begin{itemize}
\item Apply Type Theory-based verification to a problem understood by a
  wide audience
\item Find a smooth path in a maze
\item Decompose the problem
\begin{itemize}
\item Find a discrete approximation of the problem
\item Construct a piece-wise linear path
\item smoothen the angles
\end{itemize}
\item Prove the correctness of the algorithm
\begin{itemize}
\item Safety: absence of collision
\item work in progress
\end{itemize}
\end{itemize}
\end{frame}
\begin{frame}
\frametitle{Example}
\includegraphics[trim={0 0 0 \topcrop}, clip, width=\textwidth]{empty_spiral.pdf}
\end{frame}
\begin{frame}
\frametitle{Cell decomposition}
\begin{itemize}
\item Decompose the space into simple cells
\item Each cell is convex
\item Each cell is free of obstacles
\item Each cell may have safely reachable neighbors
\end{itemize}
\end{frame}
\begin{frame}
\frametitle{Vertical cell decomposition example}
\includegraphics[trim={0 0 0 \topcrop}, clip, width=\textwidth]{cells_spiral.pdf}
\end{frame}
\begin{frame}
\frametitle{Cell properties}
\begin{itemize}
\item Vertical edges are safe passages between two cells
\item Moving directly from a left-edge to a right-edge is safe
\begin{itemize}
\item and vice-versa
\end{itemize}
\item Moving from a left-edge to the cell center is safe
\begin{itemize}
\item similarly for a right-edge
\item moving from left-edge to left-edge is safe by going through the
  cell center
\end{itemize}
\end{itemize}
\end{frame}
\begin{frame}
\frametitle{Finding a path in the cell graph}
\begin{itemize}
\item A discrete path from cell to cell is found by breadth-first search
\item Connected components of the graph are defined by polygons
\item Special care for points that are already on the common edge of two cells
\item Recent improvement: take distances into account
\begin{itemize}
\item Use a graph of doors instead of cells
\item Easier to associate a distance between pairs of doors
\item Dijkstra shortest path algorithm
\end{itemize}
\item In the end, a path from door to door
\end{itemize}
\end{frame}
\begin{frame}
\frametitle{Two examples of elementary safe paths}
\includegraphics[trim={0 0 0 \topcrop}, clip, width=\textwidth]{spiral_safe2.pdf}
\end{frame}
\begin{frame}
\frametitle{piecewise linear path}
\label{broken-line}
\includegraphics[trim={0 0 0 \topcrop}, clip, width=\textwidth]{spiral_bline.pdf}
\end{frame}
\begin{frame}
\frametitle{Making corners smooth}
\begin{itemize}
\item Using quadratic Bezier curves
\item Bezier curves are given by a set of control points
   (3 for a quadratic curve)
\item Points on the curves are obtained by computing weighted barycenters
\begin{itemize}
\item The curve is enclosed in the convex hull of the control points
\end{itemize}
\item Given control points \(a_0, a_1, \ldots, a_{n-1}, a_n\), \(a_0, a_1\)
is tangent to the curve in \(a_0\)
\begin{itemize}
\item same for \(a_{n-1}, a_n\) in \(a_n\)
\end{itemize}
\end{itemize}
\end{frame}
\begin{frame}
\frametitle{Bezier curve illustration}
\begin{itemize}
\item How the point for ratio \(4/9\) is computed
\item Control points for the two subcurves are given by the new point,
the initial starting and end points, and the solid green straight edge tip
\end{itemize}
\includegraphics[trim={0 6cm 0 \topcropBezier}, clip, width=\textwidth]{bezier_example2.pdf}
\end{frame}
\begin{frame}
\frametitle{Using Bezier curves for smoothing}
\begin{itemize}
\item Add extra points in the middle of each straight line segment
\item Uses these extra points as first and last control points for Bezier curves
\item Use the angle point as the middle control point
\item Check the Bezier curve for collision and repair if need be
\end{itemize}
\end{frame}
\begin{frame}
\frametitle{Collision checking, graphically}
\includegraphics[trim={0 4cm 0 17cm}, clip, width=\textwidth]{collision.pdf}
\end{frame}
\begin{frame}
\frametitle{Not passing in the top door}
\includegraphics[trim={0 4cm 0 17cm}, clip, width=\textwidth]{collision2.pdf}
\end{frame}
\begin{frame}
\frametitle{Final trajectories}
\label{final-spiral}
\includegraphics[trim={0 0 0 \topcrop}, clip, width=\textwidth]{smooth_spiral2.pdf}
\end{frame}
\begin{frame}
\frametitle{Proof tools}
\begin{itemize}
\item Convex hulls (Pichardie \& B. 2001)
\begin{itemize}
\item Orientation predicate
\item point below or above edge
\end{itemize}
\item Linear arithmetic
\begin{itemize}
\item Algorithms only use rational numbers
\item Bezier curve intersections rely on algebraic numbers
\end{itemize}
\item Convex spaces and Bezier Curve
\begin{itemize}
\item Internship by Q. Vermande
\item Using {\tt infotheo}, especially convex and conical spaces
         (Affeldt \& Garrigue \& Saikawa 2020)
\end{itemize}
\end{itemize}
\end{frame}
\begin{frame}
\frametitle{Vertical cell decomposition proofs}
\begin{itemize}
\item Use of semi-closed vertical cells
\item Show disjoint property
\item Show that obstacles are covered by cell tops
\end{itemize}
\end{frame}
\begin{frame}
\frametitle{Future work}
\begin{itemize}
\item Intend to prove only safety
\item Produce concrete code from abstract models
\begin{itemize}
\item Move from exact computation to approximations
\item Efficient implementation of graphs
\end{itemize}
\item Already usable in \textcolor{blue}{\href{https://stamp.gitlabpages.inria.fr/trajectories.html}{web-base demonstration}.}
\begin{itemize}
\item Extracted code to Ocaml, then compile to JavaScript
\end{itemize}
\end{itemize}
\end{frame}
\end{document}


%%% Local Variables: 
%%% mode: latex
%%% TeX-master: t
%%% End: 
